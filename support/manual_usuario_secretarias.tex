\documentclass[12pt,a4paper]{article}
\usepackage[utf8]{inputenc}
\usepackage[spanish]{babel}
%\usepackage{fullpage}

\title{Manual de Usuario para Partes Diarios \\ Sección: Usuarios}
\begin{document}
\date{\today}
\maketitle

\section{Introduccion}

El siguiente manual de usuario tiene como objetivo ayudar a los usuarios (secretarias) del sistema de Partes Diarios, del Colegio Jesús en el Huerto de los Olivos, a poder utilizar correctamente el mismo. \\
De ésta manera, la información que se pueda extraer del mismo podrá ser utilizar por los directivos de forma confiable y servirá, también, para el mejor funcionamiento de la institución.

\section{Notificaciones}
Las notificaciones son una parte importante del sistema, ya que dan aviso a los usuarios, de una forma rápida y visual, de las acciones realizadas (u omitidas) por los mismos a la hora de ingresar los partes. \\
En general, las notificaciones son diarias (es decir, las mismas se actulizan en la madrugada de cada día) y entre otras, dan aviso a los secretarios si:
\begin{itemize}
	\item No se ingreso el parte diario del dia anterior
	\item Se ingresó un accidente de un Docente, pero no se reportó aviso a ART.
	\item Un parte fue rechazado o esta pendiente de corroboracion.
\end{itemize}

\subsection{Duración de las notificaciones}
Las notificaciones serán visibles durante 7 días (contando los no hábiles), a excepción de que el usuario considere que dicha notificación ya no es válida. \\
En dicho caso, el mismo podrá proceder a eliminarla, haciendo click en  la cruz que aparece en la parte superir derecha de cada notificación. Tener en cuenta que el sistema no pedirá confirmación de ésta acción, por lo que se deberá ser cauto a la hora de borrar las notificaciones y corroborar si efectivamente ésta ya no es válida.

\section{Partes ingresados}
Por otra parte, el usuario tiene la posibilidad de ver los detalles de los partes ingresados y tenerlos listos para imprimir. \\
Buscando el parte que se desea ver dentro de la tabla de partes, el usuario podra hacer click en el parte deseado y, automaticamente, se le descargara el detalle del parte en formato PDF. Si asi lo deseea, se podra imprimir o bien, tenerlo solo para control personal en su propia computadora.

\section{Ingresar Parte}
La tarea mas importante de las secretarias dentro del sistema, es la de ingresar los partes diarios correctamente. \\
Para el correcto ingreso del parte, es necesario ingresar la fecha del parte, y seleccionar la secretaria que esta ingresando el parte (para el caso en que la seccion tenga mas de una secretaria habilitada para el ingreso de partes). Luego, los pasos a seguir dependen de los datos que tenga el parte. 
\subsection{Ingresar partes}
Para agregar docentes a los partes, es necesario hacer click en "Agregar Docente". Al hacerlo, se agregara automaticamente una fila con varios campos que el usuario debera llenar. En particular, en el campo \textit{Docente}, el usuario debe escribir el apellido del docente y, de la lista desplegable que aparecer, elegir al mismo, cuidando que coincida el docente con el cargo seleccionado (ya que puede ser que varios docentes esten mas de una vez, pero con distintos cargos). \\
Para el caso de las \textit{Horas extras}, la mecanica es similar. Por ultimo, para los \textit{Accidentes} es practicamente lo mismo, solo que en este caso tambien puede haber accidentes de alumnos y, entonces, el usuario debera espcificar de que tipo se trata. \\
En caso de ser un accidente docente, es necesario dar aviso a la ART. Si no se realizo dicha accion, es necesario dejar destildado el campo de \textit{Reporte a ART}. Caso contrario, el usuario debe marcarlo ya que sino, se le dara un aviso erroneo al administrador.
\subsection{Verificar los datos ingresados}
Una vez que el usuario ingreso todos los datos necesarios y correspondientes, y luego de darle click al boton de confirmar, el sistema le mostrara en la pantalla un resumen de todos los datos ingresados de forma tal que pueda verificar que no tuvo ningun error a la hora del ingreso. \\
Si el usuario detectara algun error en el mismo, podra hacer click en el boton \textit{Modificar Parte} para volver a la pantalla anterior y, de esa forma, modificar los datos que hayan sido ingresados incorrectamente. \\
Una vez que el usuario verificara que dichos datos son correctos, podra apretar el boton de \textit{Ingresar Parte} y, de esta forma, el parte quedara ingresado en la Base de datos del sistema.

\section{Ayuda}
En esta seccion, basicamente, el usuario puede encontrar este manual de usuario, donde se explica todo lo relacionado con el sistema de partes diarios de la institucion, ademas de una seccion especial para reportar errores en el sistema, en caso de encontrarlos.
\subsection{Formulario de reporte de errores}
Como todo sistema nuevo, es posible que el sistema posea algun tipo de error que no pudo ser detectado durante la etapa de prueba. \\
Para casos como esos, es que el usuario posee un formulario de contacto donde podra enviar a los desarrolladores del sitio, el error que le ocurrio, tratando de ser lo mas claro posible para poder detectarlo y corregirlo.
\end{document}
